\documentclass[12pt]{article}

\usepackage{graphicx}
\usepackage{subfig}
\usepackage{amsmath}
\usepackage[margin=1.0in]{geometry}
\setlength{\parindent}{0pt}
\newcommand{\forceindent}{\leavevmode{\parindent=2em\indent}}

\begin{document}

\section{Introduction}

\forceindent In regards to the analytical iceberg drift model created 
by Till Wagner in 2016, the naive evaluations of complicated 
expressions and the use of single precision input data causes 
inaccurate results. This document shows how we fixed this problem by 
doing two things: one, changing the input data into double precision; 
two, doing Taylor expansions of $\alpha$ and $\beta$ for further 
evaluation using Horner's rule.
    
\section{Improvements to $\alpha$ and $\beta$ Functions}
The functions $\alpha$ and $\beta$ are defined in Dr. Wagner's paper
as:

    \begin{equation}
        \alpha \equiv \frac{1}{2\Lambda^3}(1-\sqrt{1+4\Lambda^4})
    \end{equation} 

    \begin{equation}
        \beta \equiv \frac{1}{\sqrt{2}\Lambda^3}[(1+\Lambda^4)
        \sqrt{1+4\Lambda^4} -3\Lambda^4-1]^{1/2}
    \end{equation}

where $\Lambda$ is some constant that is occasionally very small (but
never negative). 

    \begin{figure}[h]
        \centering
        \subfloat[Naive $\alpha$ function for small $u$.]
            {{\includegraphics[width=0.5\textwidth]
            {naive_alpha_plot.png}}}
        \subfloat[Naive $\beta$ function for small $u$]
            {{\includegraphics[width=0.5\textwidth]
            {naive_beta_plot.png}}}
        \caption{Evaluating naive implementations of $\alpha$ and 
                $\beta$ for small $u$.}
        \label{fig:naive}
    \end{figure}

Hence the problem herein comes from computing the following 
differences: 

    \begin{equation}
        1 - \sqrt{1+\Lambda^4}
    \end{equation}

for $\alpha$ and:

    \begin{equation}
       (1+\Lambda^4)\sqrt{1+4\Lambda^4} -3\Lambda^4-1
    \end{equation}


for $\beta$, as $\Lambda$ approaches zero. This is due to round-off 
error.\bigskip

To fix this issue, we began by taking the Taylor approximation:

    \begin{equation}
        \sqrt{1+x} = 1 + \frac{x}{2} + \sum_{k=2}^{\infty} 
                    \frac{(-1)^{k+1}(2k-3)!}{2^{2k-2}(k-2)!k!}x^k
    \end{equation} 

to reduce round-off errors in the square root.
\bigskip
Next we applied our approximation to $\alpha$:

    \begin{equation}
        \frac{\sqrt{2}}{2}(-u+\frac{u^5}{4}-\frac{u^9}{8}+
        \frac{5u^{13}}{64}-\frac{7u^{17}}{128}+O(u^{21}))
    \end{equation}

and to $\beta$:

    \begin{multline}
        \frac{\sqrt{2}}{4}(u^3-\frac{3u^7}{8}+\frac{27u^{11}}{128}-
        \frac{143u^{15}}{1024}+\frac{3315u^{19}}{32768} - 
        \frac{20349u^{23}}{262144}+ \frac{260015u^{27}}{4194304} - \\ 
        \frac{1710855u^{31}}{33554432} + 
        \frac{92116035u^{35}}{2147483648} - 
        \frac{744762895u^{39}}{17179869184} + O(u^{43}))
    \end{multline}

where the new equation for $\alpha$ contains less terms than the new
equation for $\beta$ because $\beta$ is more ill-conditioned.\bigskip

In order to avoid computing large exponents of $u$ we simplified each
new equation using Horner's rule. Therefore $\alpha$ then became:

    \begin{multline}
        u(u^4(u^4(u^4(-0.0386699020961393u^4 + 0.055242717280199) - \\ 
        0.0883883476483184) + 0.176776695296637) - 0.707106781186548)
    \end{multline}

and $\beta$ became:

    \begin{multline}
        u^3(u^4(u^4(u^4(u^4(u^4(u^4(u^4(u^4(u^4(-0.0138698305427678u^4 + \\
        0.0129890788831978) - 0.0151656272365985) + \\ 
        0.0180267866272764) + 0.0219176256311202) - \\
        0.0274446790511418) + 0.0357675015202851) - \\
        0.0493731785691779) + 0.0745776683282687) - \\
        0.132582521472478) + 0.353553390593274)
    \end{multline}

which is beneficial because the largest power of $u$ that must be 
computed is the fourth power. \clearpage%\bigskip

Now, if we look at the absolute difference between our naive 
implementations and our new Taylor approximated - Horner evaluated 
functions, we can see the difference in how they behave.

    \begin{figure}[h]
        \centering
        \subfloat[Absolute difference for $\alpha$ function.]
            {{\includegraphics[width=0.5\textwidth]
            {absolute_naive_vs_horner_alpha_plot}}}
        \subfloat[Absolute difference for $\beta$ function.]
            {{\includegraphics[width=0.5\textwidth]
            {absolute_naive_vs_horner_beta_plot}}}
        \caption{Evaluating the absolute difference of naive 
                and Horner implemtations of $\alpha$ and $\beta$ for 
                small $u$.}
        \label{fig:absolute_difference}
    \end{figure}

\clearpage
\section{Single vs. Double Precision Input Data}
In Dr. Wagner's model, the input data for sea surface temperature is in
single precision. By simply changing the importation of this data into
double precision we solved the problem of errors occurring at the 
epsilon for single precision by making the errors occur at the epsilon
for double precision.

\end{document}
